\PassOptionsToPackage{unicode=true}{hyperref} % options for packages loaded elsewhere
\PassOptionsToPackage{hyphens}{url}
%
\documentclass[]{article}
\usepackage{lmodern}
\usepackage{amssymb,amsmath}
\usepackage{ifxetex,ifluatex}
\usepackage{fixltx2e} % provides \textsubscript
\ifnum 0\ifxetex 1\fi\ifluatex 1\fi=0 % if pdftex
  \usepackage[T1]{fontenc}
  \usepackage[utf8]{inputenc}
  \usepackage{textcomp} % provides euro and other symbols
\else % if luatex or xelatex
  \usepackage{unicode-math}
  \defaultfontfeatures{Ligatures=TeX,Scale=MatchLowercase}
\fi
% use upquote if available, for straight quotes in verbatim environments
\IfFileExists{upquote.sty}{\usepackage{upquote}}{}
% use microtype if available
\IfFileExists{microtype.sty}{%
\usepackage[]{microtype}
\UseMicrotypeSet[protrusion]{basicmath} % disable protrusion for tt fonts
}{}
\IfFileExists{parskip.sty}{%
\usepackage{parskip}
}{% else
\setlength{\parindent}{0pt}
\setlength{\parskip}{6pt plus 2pt minus 1pt}
}
\usepackage{hyperref}
\hypersetup{
            pdfborder={0 0 0},
            breaklinks=true}
\urlstyle{same}  % don't use monospace font for urls
\setlength{\emergencystretch}{3em}  % prevent overfull lines
\providecommand{\tightlist}{%
  \setlength{\itemsep}{0pt}\setlength{\parskip}{0pt}}
\setcounter{secnumdepth}{0}
% Redefines (sub)paragraphs to behave more like sections
\ifx\paragraph\undefined\else
\let\oldparagraph\paragraph
\renewcommand{\paragraph}[1]{\oldparagraph{#1}\mbox{}}
\fi
\ifx\subparagraph\undefined\else
\let\oldsubparagraph\subparagraph
\renewcommand{\subparagraph}[1]{\oldsubparagraph{#1}\mbox{}}
\fi

% set default figure placement to htbp
\makeatletter
\def\fps@figure{htbp}
\makeatother
%page settings%
\usepackage[a4paper, total={6in, 8in}]{geometry}

\begin{document}

\hypertarget{juliuxe1n-reynoso}{%
\section{Julián Reynoso\footnote{julianreynoso@unc.edu.ar}}\label{juliuxe1n-reynoso}}


\hypertarget{educaciuxf3n}{%
\subsection{Educación}\label{educaciuxf3n}}

\texttt{2005-} \textbf{Licenciatura en Filosofía, Facultad de Filosofías
y Humanidades. Universidad Nacional de Córdoba}

\texttt{1997-2004} \textbf{Escuela Superior de Comercio ``Manuel
Belgrano''} Bachiller en Economía y gestión de las organizaciones y
Técnico básico en gestión organizacional.

\hypertarget{cursos-y-talleres}{%
\subsubsection{Cursos y talleres}\label{cursos-y-talleres}}

\hypertarget{talleres-dictados}{%
\paragraph{Talleres dictados}\label{talleres-dictados}}

\texttt{2019} \emph{Técnicamente, estamos escribiendo: Recursos
informáticos e Investigación en Humanidades} en el marco de las ``II
Jornadas Jóvenes Investigador*s: Producir, inventar, comunicar
saber(es)``, Centro de Investigaciones de la Facultad de Filosofía y
Humanidades; UNC.

\texttt{2018} \emph{Cómo escribí(mos) un paper? El uso del software para
sobrellevar la vida académica} en el marco de las ``I Jornadas de
Jóvenes Investigador*s: El oficio de Investigar'', Centro de
Investigaciones de la Facultad de Filosofía y Humanidades; UNC.

\texttt{2009} Colaborador en la organización e implementación del Taller
de preparación de exámenes, destinado a alumnos de primer año, dictado
por la Secretaría Académica de la Facultad.

\hypertarget{talleres-y-seminarios-tomados}{%
\paragraph{Talleres y seminarios
tomados}\label{talleres-y-seminarios-tomados}}

\texttt{2015} \textbf{Python como Herramienta para la Ingeniería},
dictado por el Ing. Martín Gaitán. Escuela de Ingeniería en Computación
de la Facultad de Ciencias Exactas, Físicas y Naturales.

\texttt{2013} \textbf{Jornadas-Taller de Presentación de Contenidos en
Entornos Virtuales de Aprendizaje}, dictado por el Programa de Educación
a Distancia de la Secretaría de Asuntos Académicos, UNC.

\texttt{2011} Representante tutor en las \textbf{Primeras Jornadas de
Tutorías Interinstitucionales}, realizadas en el Pabellón Residencial de
la FFyH, UNC.

Asistente al \textbf{Taller de actualización docente sobre Evolución},
dictado en la Facultad de Ciencias Exactas, Físicas y Naturales de la
UNC.

Curso \textbf{Herramientas computacionales para la investigación en
Filosofía, Humanidades y Ciencias Sociales} dictado por el Dr.~José
Ahumada y el Dr.~Hernán Severgnini del Área de Filosofía del CIFFyH.

\texttt{2010} \textbf{Curso-Taller Software Libre en la FFyH: opciones
éticas, pedagógicas y técnicas}. Dictado por Valeria Chervin y otros en
el marco del Proyecto ``Universidad en la Sociedad del Conocimiento:
Fortalecimiento institucional de áreas dedicadas a la enseñanza
universitaria con nuevas tecnologías''.

\hypertarget{idiomas}{%
\paragraph{Idiomas}\label{idiomas}}

\texttt{2010} Curso de 4 años de Inglés, certificado por el Departamento
Cultural de la Facultad de Lenguas de la Universidad Nacional de
Córdoba.

\hypertarget{publicaciones}{%
\subsection{Publicaciones}\label{publicaciones}}

\hypertarget{capuxedtulos-de-libro}{%
\subsubsection{Capítulos de libro}\label{capuxedtulos-de-libro}}

\texttt{2018} Ilcic, Andrés; Reynoso, Julián. Hacia una articulación de
modelos: el caso de big data. En \emph{Filosofía e historia de la
ciencia en el Cono Sur : selección de trabajos del X Encuentro de la
Asociación de Filosofía e Historia de la Ciencia del Cono Sur}. Silvio
Seno Chibeni et. al (Editores). ISBN: 978-950-33-1401-2. P. 181-193.

\texttt{2015} Ilcic, Andrés; Reynoso, Julián. Si de entender se trata:
el rol de la visualización en el procesamiento de datos. En
\emph{Epistemología y prácticas científicas}. Víctor Rodriguez, Marisa
Velasco, Pío García (compiladores). ISBN 978-987-707-010-1. P. 171-185.

\hypertarget{presentaciones-en-congresos}{%
\subsubsection{Presentaciones en
congresos}\label{presentaciones-en-congresos}}

\texttt{2019} ``Apuntes sobre confiabilidad en predicciones de cambio
climático''. Presentado en las 1ras Jornadas de Jóvenes Investigadores
en Filosofía de la Ciencia. Córdoba.

\texttt{2018} ``Y entonces ¿cómo va a estar el clima? Algunas notas
sobre la robustez para la predicción de sistemas complejos'', con Andrés
A. Ilcic para la mesa \emph{¿Y que hacen por nosotros las simulaciones
computacionales y los modelos? Apuntes sobre robustez, representación y
observabilidad.} Presentado en las XIX Jornadas de Epistemología e
Historia de la Ciencia. Córdoba Capital, Córdoba.

\texttt{2018} ``El paper científico en la época de la reproductibilidad
técnica'', con Andrés A. Ilcic en el \emph{XI Encuentro Nacional de
Filosofía}. Córdoba Capital, Córdoba.

\texttt{2018} ``Sobre la complejidad de la restauración y la
restauración de la complejidad'', con Cecilia Estrabou para el simposio
\emph{''Filosofía, Historia y Enseñanza de las Ciencias''} en el XI
Encuentro de la Asociación de Historia y Filosofía de las Ciencias del
Cono Sur. Buenos Aires, Argentina.

``La visualización científica como instrumento para la enseñanza de la
ciencia'' para el simposio \emph{''Filosofía, Historia y Enseñanza de
las Ciencias''} en el XI Encuentro de la Asociación de Historia y
Filosofía de las Ciencias del cono sur. Buenos Aires, Argentina.

\texttt{2016} ``El nuevo problema de la vieja evidencia'' en el X
Encuentro de Filosofía e Historia de la Ciencia, organizado por AFHIC.
Aguas Do Lindoia, Brasil.

``Hacia una articulación de modelos: el caso de big data'' con Andrés
Ilcic en el X Encuentro de Filosofía e Historia de la Ciencia,
organizado por AFHIC. Aguas Do Lindoia, Brasil.

\texttt{2015} ``El crimen del doblecontar en las prácticas científicas''
en la Mesa Evidencia robustez y (ab)uso de datos. Epistemología de las
prácticas científicas, presentado en las XXVI Jornadas de Epistemología
e Historia de la Ciencia.

\hypertarget{antecedentes-en-investigaciuxf3n}{%
\subsection{Antecedentes en
investigación}\label{antecedentes-en-investigaciuxf3n}}

\hypertarget{grupos-de-investigaciuxf3n}{%
\subsubsection{Grupos de
investigación}\label{grupos-de-investigaciuxf3n}}

\hypertarget{integrante}{%
\paragraph{Integrante}\label{integrante}}

\texttt{2018-Presente} PICT FONCyT.

\texttt{2016-2018} Miembro del proyecto bianual tipo A ``Simulación,
explicación y experimentación: una aproximación'' financiado por SECyT,
UNC; dirigido por la Dra. Marisa Velasco.

%\#\# Antecedentes docentes

\hypertarget{antecedentes-en-docencia}{%
\subsection{Antecedentes en
docencia}\label{antecedentes-en-docencia}}

\hypertarget{tutoruxedas}{%
\subsubsection{Tutorías}\label{tutoruxedas}}

\texttt{2013} Tutor estudiantil becado en el proyecto ``Programa de
Apoyo y Mejoramiento de Enseñanza de Grado'', según resolución
N°160/2013.

\texttt{2011-2012} Tutor estudiantil becado en el proyecto ``Programa de
Apoyo y Mejoramiento de Enseñanza de Grado'', según resolución N°1542.

\hypertarget{extensiuxf3n}{%
\subsection{Extensión}\label{extensiuxf3n}}

\texttt{2019} Participación en ``Filopalooza: un festival de
filosofía''. Ciudad Autónoma de Buenos Aires.

\texttt{2018} Tutor en el Hackatón Desafíos Científicos 2018, organizado
por el Ministerio de Educación de la Ciudad Autónoma de Buenos Aires.

\texttt{2017} Tutor en el Hackatón Desafíos Científicos 2017, organizado
por el Ministerio de Educación de la Ciudad Autónoma de Buenos Aires.

\texttt{2016} Tutor en el Hackatón Desafíos Científicos 2016, organizado
por NASA Space Apps y el Ministerio de Educación de la Ciudad Autónoma
de Buenos Aires.

\texttt{2012-2013} Implementación y mantenimiento de aulas virtuales en
plataforma Moodle para difusión del proyecto ``Problemáticas
relacionadas con el paquete transgénico del cultivo de soja, con
especial consideración al impacto sobre el agua'', financiado por el
Ministerio de Desarrollo Social de la Nación.

\hypertarget{otros-antecedentes-laborales}{%
\subsection{Otros antecedentes
laborales}\label{otros-antecedentes-laborales}}

\hypertarget{revistas-acaduxe9micas}{%
\subsubsection{Revistas académicas}\label{revistas-acaduxe9micas}}

\texttt{2016\ –\ 2017} Secretario administrativo de la Revista
Epistemología e Historia de la Ciencia.

\hypertarget{congresos}{%
\subsubsection{Congresos}\label{congresos}}

\texttt{2013\ –\ Presente} Secretario durante las Jornadas de
Epistemología e Historia de la Ciencia.

\end{document}
