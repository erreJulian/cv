\PassOptionsToPackage{unicode=true}{hyperref} % options for packages loaded elsewhere
\PassOptionsToPackage{hyphens}{url}
%
\documentclass[]{article}
\usepackage{lmodern}
\usepackage{amssymb,amsmath}
\usepackage{ifxetex,ifluatex}
\usepackage{fixltx2e} % provides \textsubscript
\ifnum 0\ifxetex 1\fi\ifluatex 1\fi=0 % if pdftex
  \usepackage[T1]{fontenc}
  \usepackage[utf8]{inputenc}
  \usepackage{textcomp} % provides euro and other symbols
\else % if luatex or xelatex
  \usepackage{unicode-math}
  \defaultfontfeatures{Ligatures=TeX,Scale=MatchLowercase}
\fi
% use upquote if available, for straight quotes in verbatim environments
\IfFileExists{upquote.sty}{\usepackage{upquote}}{}
% use microtype if available
\IfFileExists{microtype.sty}{%
\usepackage[]{microtype}
\UseMicrotypeSet[protrusion]{basicmath} % disable protrusion for tt fonts
}{}
\IfFileExists{parskip.sty}{%
\usepackage{parskip}
}{% else
\setlength{\parindent}{0pt}
\setlength{\parskip}{6pt plus 2pt minus 1pt}
}
\usepackage{hyperref}
\hypersetup{
            pdfborder={0 0 0},
            breaklinks=true}
\urlstyle{same}  % don't use monospace font for urls
\setlength{\emergencystretch}{3em}  % prevent overfull lines
\providecommand{\tightlist}{%
  \setlength{\itemsep}{0pt}\setlength{\parskip}{0pt}}
\setcounter{secnumdepth}{0}
% Redefines (sub)paragraphs to behave more like sections
\ifx\paragraph\undefined\else
\let\oldparagraph\paragraph
\renewcommand{\paragraph}[1]{\oldparagraph{#1}\mbox{}}
\fi
\ifx\subparagraph\undefined\else
\let\oldsubparagraph\subparagraph
\renewcommand{\subparagraph}[1]{\oldsubparagraph{#1}\mbox{}}
\fi

% set default figure placement to htbp
\makeatletter
\def\fps@figure{htbp}
\makeatother


\date{}

\begin{document}

\hypertarget{juliuxe1n-reynoso}{%
\section{Julián Reynoso}\label{juliuxe1n-reynoso}}

\leavevmode\hypertarget{webaddress}{}%
julianreynoso@unc.edu.ar

\hypertarget{en-la-actualidad}{%
\subsection{En la actualidad}\label{en-la-actualidad}}

Escribo mi tesis de licenciatura en filosofía y participo de los
proyectos de investigación ``Simulaciones computacionales y
experimentación desde la perspectiva de las prácticas científicas: una
aproximación'' PICT 2016-1524, financiado por FONCyT; y ``Modelar,
simular y experimentar: un análisis epistemológico desde las prácticas
científicas'' financiado por SECyT de la Universidad Nacional de
Córdoba.

\hypertarget{intereses-de-investigaciuxf3n}{%
\subsubsection{Intereses de
investigación}\label{intereses-de-investigaciuxf3n}}

Filosofía de las simulaciones computacionales,filosofía de las ciencias,
Epistemología, modelización, visualización, modelos de cambio climático.

\hypertarget{habilidades-y-aptitudes}{%
\subsubsection{Habilidades y aptitudes}\label{habilidades-y-aptitudes}}

En el marco de mi participación en las investigaciones arriba
mencionadas comencé a indagar sobre métodos y formas de agilizar la labor de la investigación y empecé a probar distintas
herramientas, desde gestores bibliográficos hasta LaTeX, pasando por
Atlas.Ti y distintos gestores de proyectos y tareas. Además, gracias a
mi paso como secretario de la revista ``Epistemología e Historia de la
Ciencia'', tengo experiencia con Open Journal System.

\begin{itemize}
\tightlist
\item
  Office (Word, PowerPoint y Excel).
\item
  LaTeX.
\item
  Gestores de bibliografía (Zotero, JabRef y Mendeley).
\item
  Bases de datos para búsquedas académicas (Google Scholar, EBSCO y
  PhilPapers).
\item
  HTML y CSS (principiante).
\item
  Open Journal System (sistema de gestión de revistas académicas).
\end{itemize}

\hypertarget{educaciuxf3n}{%
\subsection{Educación}\label{educaciuxf3n}}

\texttt{2005-} \textbf{Licenciatura en Filosofía, Facultad de Filosofías
y Humanidades. Universidad Nacional de Córdoba}

\texttt{1997-2004} \textbf{Escuela Superior de Comercio ``Manuel
Belgrano''} Bachiller en Economía y gestión de las organizaciones y
Técnico básico en gestión organizacional.

\hypertarget{cursos-y-talleres}{%
\subsubsection{Cursos y talleres}\label{cursos-y-talleres}}

\hypertarget{talleres-dictados}{%
\paragraph{Talleres dictados}\label{talleres-dictados}}

\texttt{2019} \emph{Técnicamente, estamos escribiendo: Recursos
informáticos e Investigación en Humanidades} en el marco de las ``II
Jornadas Jóvenes Investigador*s: Producir, inventar, comunicar
saber(es)``, Centro de Investigaciones de la Facultad de Filosofía y
Humanidades; UNC.

\texttt{2018} \emph{Cómo escribí(mos) un paper? El uso del software para
sobrellevar la vida académica} en el marco de las ``I Jornadas de
Jóvenes Investigador*s: El oficio de Investigar'', Centro de
Investigaciones de la Facultad de Filosofía y Humanidades; UNC.

\texttt{2009} Colaborador en la organización e implementación del Taller
de preparación de exámenes, destinado a alumnos de primer año, dictado
por la Secretaría Académica de la Facultad.

\hypertarget{talleres-y-seminarios-tomados}{%
\paragraph{Talleres y seminarios
tomados}\label{talleres-y-seminarios-tomados}}

\texttt{2015} \textbf{Python como Herramienta para la Ingeniería},
dictado por el Ing. Martín Gaitán. Escuela de Ingeniería en Computación
de la Facultad de Ciencias Exactas, Físicas y Naturales.

\texttt{2013} \textbf{Jornadas-Taller de Presentación de Contenidos en
Entornos Virtuales de Aprendizaje}, dictado por el Programa de Educación
a Distancia de la Secretaría de Asuntos Académicos, UNC.

\texttt{2011} Representante tutor en las \textbf{Primeras Jornadas de
Tutorías Interinstitucionales}, realizadas en el Pabellón Residencial de
la FFyH, UNC.

Asistente al \textbf{Taller de actualización docente sobre Evolución},
dictado en la Facultad de Ciencias Exactas, Físicas y Naturales de la
UNC.

Curso \textbf{Herramientas computacionales para la investigación en
Filosofía, Humanidades y Ciencias Sociales} dictado por el Dr.~José
Ahumada y el Dr.~Hernán Severgnini del Área de Filosofía del CIFFyH.

\texttt{2010} \textbf{Curso-Taller Software Libre en la FFyH: opciones
éticas, pedagógicas y técnicas}. Dictado por Valeria Chervin y otros en
el marco del Proyecto ``Universidad en la Sociedad del Conocimiento:
Fortalecimiento institucional de áreas dedicadas a la enseñanza
universitaria con nuevas tecnologías''.

\texttt{2009} \textbf{Seminario de Formación Docente para Ayudantes
Alumnos de Cátedras de Primer Año}, dictado por la Coordinadora General
del Curso de Nielación de la Secretaría Académica de la Facultad de
Filosofía y Humanidades, UNC.

\textbf{Taller Evaluación del Aprendizaje en las Ciencias Naturales},
dictado por la Dra. Amelia Gort, en el marco del Proyecto Redes
Universitarias II, dependiente de la Secretaría de Asuntos Académicos de
la UNC.

\hypertarget{idiomas}{%
\paragraph{Idiomas}\label{idiomas}}

\texttt{2010} Curso de 4 años de Inglés, certificado por el Departamento
Cultural de la Facultad de Lenguas de la Universidad Nacional de
Córdoba.

\texttt{2005} \emph{Customer Service Representative}, certificado por la
Secretaria de Extensión de la Facultad de Lenguas de la Universidad
Nacional de Córdoba.

\hypertarget{publicaciones}{%
\subsection{Publicaciones}\label{publicaciones}}

\hypertarget{capuxedtulos-de-libro}{%
\subsubsection{Capítulos de libro}\label{capuxedtulos-de-libro}}

\texttt{2018} Ilcic, Andrés; Reynoso, Julián. Hacia una articulación de
modelos: el caso de big data. En \emph{Filosofía e historia de la
ciencia en el Cono Sur : selección de trabajos del X Encuentro de la
Asociación de Filosofía e Historia de la Ciencia del Cono Sur}. Silvio
Seno Chibeni et. al (Editores). ISBN: 978-950-33-1401-2. P. 181-193.

\texttt{2015} Ilcic, Andrés; Reynoso, Julián. Si de entender se trata:
el rol de la visualización en el procesamiento de datos. En
\emph{Epistemología y prácticas científicas}. Víctor Rodriguez, Marisa
Velasco, Pío García (compiladores). ISBN 978-987-707-010-1. P. 171-185.

\texttt{2013} Reynoso, Julián. Repasando el rol de los experimentos
imaginarios en ciencia. En \emph{Epistemología e Historia de la Ciencia.
Seleccion de trabajos de las XXIII Jornadas}, volumen 19. Hernán
Severgnini, José Gustavo Morales, Diana Luz Rabinovich (editores). ISBN
978-950-331073-1. P. 350-358.

\hypertarget{presentaciones-en-congresos}{%
\subsubsection{Presentaciones en
congresos}\label{presentaciones-en-congresos}}

\texttt{2019} ``Apuntes sobre confiabilidad en predicciones de cambio
climático''. Presentado en las 1ras Jornadas de Jóvenes Investigadores
en Filosofía de la Ciencia. Córdoba.

\texttt{2018} ``Y entonces ¿cómo va a estar el clima? Algunas notas
sobre la robustez para la predicción de sistemas complejos'', con Andrés
A. Ilcic para la mesa \emph{¿Y que hacen por nosotros las simulaciones
computacionales y los modelos? Apuntes sobre robustez, representación y
observabilidad.} Presentado en las XIX Jornadas de Epistemología e
Historia de la Ciencia. Córdoba Capital, Córdoba.

\texttt{2018} ``El paper científico en la época de la reproductibilidad
técnica'', con Andrés A. Ilcic en el \emph{XI Encuentro Nacional de
Filosofía}. Córdoba Capital, Córdoba.

\texttt{2018} ``Sobre la complejidad de la restauración y la
restauración de la complejidad'', con Cecilia Estrabou para el simposio
\emph{''Filosofía, Historia y Enseñanza de las Ciencias''} en el XI
Encuentro de la Asociación de Historia y Filosofía de las Ciencias del
Cono Sur. Buenos Aires, Argentina.

``La visualización científica como instrumento para la enseñanza de la
ciencia'' para el simposio \emph{''Filosofía, Historia y Enseñanza de
las Ciencias''} en el XI Encuentro de la Asociación de Historia y
Filosofía de las Ciencias del cono sur. Buenos Aires, Argentina.

\texttt{2016} ``El nuevo problema de la vieja evidencia'' en el X
Encuentro de Filosofía e Historia de la Ciencia, organizado por AFHIC.
Aguas Do Lindoia, Brasil.

``Hacia una articulación de modelos: el caso de big data'' con Andrés
Ilcic en el X Encuentro de Filosofía e Historia de la Ciencia,
organizado por AFHIC. Aguas Do Lindoia, Brasil.

\texttt{2015} ``El crimen del doblecontar en las prácticas científicas''
en la Mesa Evidencia robustez y (ab)uso de datos. Epistemología de las
prácticas científicas, presentado en las XXVI Jornadas de Epistemología
e Historia de la Ciencia.

\texttt{2014} ``¿Qué es esa cosa llamada Big Data?'' en la mesa
\emph{Epistemología y prácticas científicas IV: Datos, resolución de
problemas y automatismo}, en las XXV Jornadas de Epistemología e
Historia de la Ciencia, IX Encuentro de AFHIC.

``Ahora me ves: La importancia de lo visual en la técnica y la técnica
de lo visual'' para el \emph{Simposio de Historia y Filosofía de la
Tecnología} en las XXV Jornadas de Epistemología e Historia de la
Ciencia, IX Encuentro de AFHIC.

\texttt{2013} ``Si de entender se trata: el rol de la visualización en
el procesamiento de datos'', en la mesa Obtención y tratamiento de
datos: Un abordaje desde la epistemología de las prácticas científicas;
en las XXIV Jornadas de Epistemología e Historia de la Ciencia; La
Falda, Córdoba.

``Persiguiendo tormentas a resguardo: Visualizando simulaciones de
tormentas para dar cuenta del fenómeno'', en coautoría con Andrés Ilcic;
en el Segundo Workshop sobre modelos y simulaciones en ciencia; SADAF;
Buenos Aires.

\texttt{2012} ``Repensando el rol de los experimentos imaginarios en
ciencia'' XXIII Jornadas de Epistemología e Historia de la Ciencia; La
Falda, Córdoba.

\texttt{2008} ``Relativismo, pragmatismo y filosofía latinoamericana:
hacia la construcción de un espacio propio'' en las IV Jornadas de
Filosofía Teóricas, FFyH, UNC.

\texttt{2007} ``W.V. Quine y la traducción radical; algunos comentarios
sobre la relatividad ontológica.'' XIV Congreso Nacional de Filosofía de
AFRA -- Tucumán

``¿Qué formación en investigación científica tienen los egresados de la
UNC en carreras de formación científica?'' Con Estrabou, Cecilia; Luti,
Ricardo; En el V Encuentro Nacional y II Latinoamericano La universidad
como objeto de investigación; Tandil, Buenos Aires.

\hypertarget{asistencia-a-congresos}{%
\subsubsection{Asistencia a congresos}\label{asistencia-a-congresos}}

\texttt{2013} I Encuentro de Filosofía de las Ciencias Cognitivas y del
Comportamiento.

\texttt{2012} Primer Workshop sobre modelos y simulaciones en ciencias.

\texttt{2011} XXII Jornadas de Epistemología e Historia de la Ciencia.
VI Workshop de Astronomía Computacional, organizado por la Asociación
Argentina de Astronomía en el Observatorio Astronómico de Córdoba.

Jornadas ``Empezar a Investigar'', dictada por el Dr.~Andrés Laguens,
organizadas por la Secretaría de Ciencia y Técnica de la Facultad de
Filosofía y Humanidades.

\texttt{2010} XXI Jornadas de Epistemología e historia de la Ciencia.

\texttt{2008} XVIII Jornadas de Epistemología e Historia de la Ciencia.

\texttt{2007} IV Jornadas de Filosofía Teórica: Conceptos, Creencias y
Racionalidad.

\texttt{2006} XVII Jornadas de Epistemología e Historia de la Ciencia.

III Jornadas de Filosofía Teórica: Conocimiento, Normatividad y Acción.

\hypertarget{antecedentes-en-investigaciuxf3n}{%
\subsection{Antecedentes en
investigación}\label{antecedentes-en-investigaciuxf3n}}

\hypertarget{grupos-de-investigaciuxf3n}{%
\subsubsection{Grupos de
investigación}\label{grupos-de-investigaciuxf3n}}

\hypertarget{integrante}{%
\paragraph{Integrante}\label{integrante}}

\texttt{2018-Presente} PICT FONCyT.

\texttt{2016-2018} Miembro del proyecto bianual tipo A ``Simulación,
explicación y experimentación: una aproximación'' financiado por SECyT,
UNC; dirigido por la Dra. Marisa Velasco.

\texttt{2014-2016} Miembro del proyecto bianual tipo A ``Simulación,
computación y experimentación desde la perspectiva de las prácticas
científicas'' financiado por SECyT, UNC; dirigido por la Dra. Marisa
Velasco.

\texttt{2011-2015} Miembro colaborador del proyecto ``Las prácticas
científicas experimentales y observacionales: enfoque epistemológico
desde las simulaciones computacionales y la modelización matemática'',
(PICT-2011-0280) dirigido por el Dr.~Víctor Rodríguez, FONCyT. Aprobado
por resolución 140-12.

\texttt{2011-2013} Miembro del proyecto bianual tipo A ``Filosofía de
las prácticas científicas: computación, simulación y experimentación''
financiado por SECyT, UNC; dirigido por el Dr.~Pío García.

Ayudante Alumno en el proyecto ``Las prácticas experimentales:
metodología y epistemología en la vida de los laboratorios'', dirigido
por el Dr.~Víctor Rodríguez. CIFFyH, UNC. Designado por resolución
247/2011; Aprobado por resolución 11/2014.

\texttt{2008-2010} Integrante del proyecto bianual tipo A ``Modelos de
inferencia no estándar, ontologías formales y sistemas multiagentes'',
SECyT, UNC, dirigido por el Dr.~Luis Urtubey. 

%\#\# Antecedentes docentes

\hypertarget{antecedentes-en-docencia}{%
\subsection{Antecedentes en
docencia}\label{antecedentes-en-docencia}}
\hypertarget{ayudantuxedas-de-cuxe1tedra}{%
\subsubsection{Ayudantías de
cátedra}\label{ayudantuxedas-de-cuxe1tedra}}

\texttt{2010-2012} Ayudante alumno, cátedra de Filosofía de la Ciencia,
Escuela de Filosofía (FFyH, UNC). Designado por resolución 168/2010 del
HCD. Aprobada por resolución 512/33.

\texttt{2009-2011} Ayudante alumno, cátedra de Seminario Metodológico,
Escuela de Filosofía (FFyH, UNC). Designado por resolución 404/09 del
HCD. Aprobada por resolución 533/12.

\texttt{2008/2010} Ayudante alumno, cátedra de Lógica I, Escuela de
Filosofía (FFyH, UNC) Designado por resolución 155/08 del HCD. Aprobada
por resolución Decanal 1049/2010.

\texttt{2007-2009} Ayudante alumno, cátedra de Seminario Metodológico,
Escuela de Filosofía (FFyH, UNC) Designado por resolución 336/07 del
HCD. Aprobado por Resolución Decanal 2150/09.

\hypertarget{tutoruxedas}{%
\subsubsection{Tutorías}\label{tutoruxedas}}

\texttt{2013} Tutor estudiantil becado en el proyecto ``Programa de
Apoyo y Mejoramiento de Enseñanza de Grado'', según resolución
N°160/2013.

\texttt{2011-2012} Tutor estudiantil becado en el proyecto ``Programa de
Apoyo y Mejoramiento de Enseñanza de Grado'', según resolución N°1542.

\hypertarget{comisiones-evaluadoras}{%
\subsubsection{Comisiones evaluadoras}\label{comisiones-evaluadoras}}

\texttt{2016} Representante estudiantil en el concurso por dos cargos de
Profesor Asistente para la cátedra de Introducción a la Problemática
Filosófica de la Escuela de Filosofía de la FFyH.

\texttt{2014} Representante estudiantil en la Comisión Evaluadora para
la Selección de Antecedentes para un cargo de Profesor Asistente en la
Cátedra de Seminario Metodológico.

\texttt{2013} Representante estudiantil en el Tribunal de Selección de
Antecedentes y Entrevista Personal para Adscriptos y Ayudantes Alumnos
en el Área de Filosofía del Centro de Investigaciones de la Facultad de
Filosofía y Humanidades, certificado el 18 de Abril de 2013.

\texttt{2012} Veedor estudiantil en el Concurso por oposición y
antecedentes para un cargo de Profesor Asistente de la Cátedra de
Seminario Metodológico. Designado por resolución 707/2011, certificado
el día 31/10/2012.

\texttt{2009} Representante estudiantil en la Selección de antecedentes
para el equipo estable del ``Programa ciclos de nivelación, seguimiento
de los primeros años y articulación con la escuela media''. Designado
por resolución 519/09; certificado el día 24/10/2009.

\hypertarget{becas-obtenidas}{%
\subsection{Becas obtenidas}\label{becas-obtenidas}}

\texttt{2013} Tutor estudiantil becado en el proyecto ``Programa de
Apoyo y Mejoramiento de Enseñanza de Grado'', según resolución
N°160/2013.

\texttt{2011-2012} Tutor estudiantil becado en el Programa de Apoyo y
Mejoramiento de la Enseñanza de Grado.

\hypertarget{extensiuxf3n}{%
\subsection{Extensión}\label{extensiuxf3n}}

\texttt{2019} Participación en ``Filopalooza: un festival de
filosofía''. Ciudad Autónoma de Buenos Aires.

\texttt{2018} Tutor en el Hackatón Desafíos Científicos 2018, organizado
por el Ministerio de Educación de la Ciudad Autónoma de Buenos Aires.

\texttt{2017} Tutor en el Hackatón Desafíos Científicos 2017, organizado
por el Ministerio de Educación de la Ciudad Autónoma de Buenos Aires.

\texttt{2016} Tutor en el Hackatón Desafíos Científicos 2016, organizado
por NASA Space Apps y el Ministerio de Educación de la Ciudad Autónoma
de Buenos Aires.

\texttt{2012-2013} Implementación y mantenimiento de aulas virtuales en
plataforma Moodle para difusión del proyecto ``Problemáticas
relacionadas con el paquete transgénico del cultivo de soja, con
especial consideración al impacto sobre el agua'', financiado por el
Ministerio de Desarrollo Social de la Nación.

\texttt{2010} Disertante en la muestra de carreras de la Universidad
Nacional de Córdoba, organizada por la Secretaría de Asuntos
Estudiantiles de dicha casa de estudios.

\texttt{2009} Disertante en la muestra de carreras de la Universidad
Nacional de Córdoba, organizada por la Secretaría de Asuntos
Estudiantiles de dicha casa de estudios.

\hypertarget{otros-antecedentes-laborales}{%
\subsection{Otros antecedentes
laborales}\label{otros-antecedentes-laborales}}

\hypertarget{revistas-acaduxe9micas}{%
\subsubsection{Revistas académicas}\label{revistas-acaduxe9micas}}

\texttt{2016\ –\ 2017} Secretario administrativo de la Revista
Epistemología e Historia de la Ciencia.

\hypertarget{congresos}{%
\subsubsection{Congresos}\label{congresos}}

\texttt{2013\ –\ Presente} Secretario durante las Jornadas de
Epistemología e Historia de la Ciencia.

\end{document}
